%
\documentclass[runningheads]{llncs}
%
\usepackage{makeidx}  % allows for indexgeneration
\usepackage{amsmath} % allows for \text{}
%
\begin{document}
%
%\frontmatter          % for the preliminaries
%
%\pagestyle{headings}  % switches on printing of running heads
%
%
%\mainmatter              % start of the contributions
%
\title{Online Outlier Detection over Large Datasets}
\titlerunning{Online Outlier Detection}  % abbreviated title (for running head)
\subtitle{Seminar - Multimedia Retrieval and Data Mining}
%
\author{Matthias Hansen}
\authorrunning{Hansen} % abbreviated author list (for running head)
%
%
\institute{Data Management and Data Exploration Group\\
RWTH Aachen University\\
Germany\\
\email{matthias.hansen@rwth-aachen.de}}

\maketitle              % typeset the title of the contribution

\begin{abstract}
Performing Distance-Based Outlier Analysis on Large Datasets poses a challenge with respect to computation time. Due to its $n\log(n)$ complexity, the kd-tree based algorithm for $k$-nearest neighbor search is not applicable for interactive outlier exploration over data sets whose sizes ranges in the millions. In this paper, we present a system that preprocesses the results of an initial application of the standard algorithm and stores them in memory to efficiently support any following outlier detection queries and also two novel outlier analysis operations that allow data analysts to gain insight into the outlier status of relevant parts of the data set much more quickly than the traditional approach of repeatedly finding $k$-nearest neighbors would.
\end{abstract}

\section{Motivation}
Finding outliers in data is a problem frequently occured in data analysis. For example credit card providers use outlier detection to find transactions that could be fraudulent. Similarly, in a network security context, outlier detection can be used to identify intruders. It has also been used for goals as diverse as data cleaning, financial planning and severe weather prediction.
%TODO: expand this paragraph

\subsection{Outlier Definitions}

There is no universally accepted precise definition of what constitutes an outlier in a data set. Instead, several different definitions are used as appropriate for the task at hand.

Traditionally, outliers have been defined statistically, i.e. with respect to a statistical distribution. In this case, every data point to which the distribution assigns a probability below some threshold is classified as an outlier. However, this is only applicable if those points that are not outliers are known to follow a specific distribution. This is not always the case and choosing a statistical distribution so that it reflects the data set is challenging and may require domain knowledge and--especially for multivariate data knowledge about statistics.

A competing outlier definition is the distance-based outlier. A point is a distance-based outlier if its $\varepsilon$-neighborhood with respect to a distance measure contains less than $k$ points, where $\varepsilon$, $k$ and the distance measure can be chosen as necessary for the task at hand.

\begin{definition}[Distance-Based Outlier]
    Let DB be a set of tuples. A data point $p\in DB$ is an outlier with respect to a parameter setting $(\varepsilon,k)$ iff 
    $|\{q \in DB | d(p,q) < \varepsilon\}| < k$.

    $O_{(\varepsilon,k)}(DB) := \{p\in DB \;| \;p\;\text{ is a distance-based outlier w.r.t. }(\varepsilon,k) \}$
\end{definition}

This definition is impartial to the distribution of the data and it is especially useful for geospatial data, where the Euclidean distance has additional meaning apart from indicating (dis-)similarity. Drawbacks of this definition include inability to cope with arbitrarily high-dimensional data, since the distances of the most-distant and the least-distant points approach each other as the dimensionality of data set grows--a phenomenon referred to as the curse of dimensionality. %TODO cite that one paper about curse of dimensionality.
%TODO Improve this paragraph?

\subsection{State of the Art}

The naive way to detect distance-based outliers is to simply compute the distances between each pair of points. From the resulting distance matrix, the k-th closest other point for every point $p_i$ can be found by scanning $p_i$'s row in the distance matrix. Then one has to check whether the distance to this k-th closest point is bigger than $\varepsilon$. If this is the case, then $p_i$ is an outlier.

The time and space complexities of this algorithm are both in $\mathcal{O}((n\cdot d)^2)$ ($n$ being the number of rows and $d$ the number of columns in the DB), so the algorithm does not scale well for large amounts of data.

Faster algorithms to solve this problem are usually based on spatial index structures. For example, a kd-tree based algorithm exists that essentially solves this problem in $\mathcal{O}(n\cdot d \cdot\log(n\cdot d))$, by first finding the $k$-nearest neighbors of each $p_i$ and then comparing the distance to the $k$-th nearest neighbor to $\varepsilon$ to find out whether $p_i$ is an outlier. 

This algorithm clearly performs better for large datasets. However in our experiments we found that for e.g. a two-dimensional dataset with . %TODO insert values here

This is not a big problem if only one isolated outlier detection query has to be performed. However, a typical use case for outlier detection is exploratory data analysis, where it is not clear ahead of time whether one set of parameters ($k$,$\varepsilon$) produces the desired result. An analyst might, for example, run an outlier detection query with a value of $k$ that is very high, wait a long time for the result, and then find out that the outlier detection algorithm has classified every point as an outlier. They would then have to re-run the outlier detection algorithm with a different set of parameters, completely recomputing the $k$-nearest neighbors in the process.
%TODO Add experimental results here
%TODO: Also discuss the DOLPHIN System.
\subsection{Considerations for a New Outlier Detection System}
As mentioned before, common distance-based outlier algorithms recompute a k-nearest neighbors (knn) query on every outlier detection request. However, one knn result can be used repeatedly because querying for e.g. the 5 nearest neighbors also reveals the fourth, third, second etc. nearest neighbor and thus outliers can be detected for any parameter setting $(\varepsilon,k^\prime)$ where $k^\prime \leq k$ in linear time simply by storing the knn result.

If it is known ahead of time that certain parameter combinations are not of interest, then part of the knn result can be discarded while still allowing for this linear outlier detection. For instance, if it is known that outlier detection queries with $k < 3$ are not ``useful'' for a dataset, then all information about first and second nearest neighbors can be deleted. Similarly, if maximum and minimum ``useful'' values for $\varepsilon$ are known then all points whose closest neighbor is above the maximum $\varepsilon$ are clearly already outliers for any ``useful'' parameter setting. Nearest neighbor information on these points can thus also be discarded and they can be regarded \emph{const outliers} with respect to the ``useful'' parameter settings. The same procedure can be applied where even the distance of the k-th nearest neighbor is above the minimum useful value for $\varepsilon$ and these values can be classified as \emph{const inliers}. If many points are either \emph{const inliers} or \emph{const outliers}, then outlier detection has to merely consider the remaining points, the \emph{outlier candidates}, greatly decreasing both memory usage and processing time.

Additionally, the traditional outlier detection query is not the only conceivable mode of gaining information about outliers from a data set. We propose two other interesting problems that a new outlier detection should be able to solve:

\begin{definition}[Comparative Outlier Analytics (CO)]

\noindent Given $P\subseteq DB$, find maximal $Q\subseteq DB$, such that the following proposition holds:

\noindent $\forall (\varepsilon,k) P \subseteq O_{(\varepsilon, k)}(DB) \implies \forall (\varepsilon,k)Q\subseteq O_{(\varepsilon, k)}(DB)$
\end{definition}

\begin{definition}[Outlier-Specific Parameter Space Exploration (PSE)]

\noindent Given $P\subseteq DB$ and $\delta \in (-1,1)$, identify all parameter settings $(k,\varepsilon)$, such that

\noindent(1) $|O_{(k,\varepsilon)}(DB)| = (1 - \delta) \cdot |P|$

\noindent(2) $O_{(k,\varepsilon)}(DB) \subseteq P$ if $\delta \geq 0$ or $P \subseteq O_{(k,\varepsilon)}(DB)$ if $\delta \leq 0$
\end{definition}

Algorithms to solve these problems would allow analysts to use knowledge about the outlier status of certain points in order to find all outliers. For instance, in the credit card fraud detection scenario, it could be known to an them that certain transactions were fraudulent and these transactions could then be used in a CO query to find other transactions that might be fraudulent.

Similarly, the analyst might also know from a statistic maintained by their business that a certain percentage of transactions is fraudulent. They could apply PSE to a set of known fraudulent transactions and set delta such that the returned parameter settings produce outlier sets with a size of that percentage compared to the data set.

To conclude, a new outlier detection system should take ranges instead of fixed values for $\varepsilon$ and $k$ and use these to support efficient outlier detection queries within these ranges as well as the other outlier operations we presented.
\section{Overview of the ONION System}
The ONION system addresses the considerations of the previous section by splitting the Outlier Analysis task into an offline and an online phase. 

In the offline phase, an initial knn query is performed and the result is processed into a series of three data structures: \emph{O-Space}, \emph{P-Space} and \emph{D-Space}, each of them adding to computation time.

In the online phase, Outlier Detection, Comparative Outlier Analytics and Outlier-Centric Parameter Space Exploration queries can be issued by the user. These queries can be processed by consulting the data structures computed in the offline phase without referring to the original data set.

The ONION system takes its name from this \underline{on}line outlier detect\underline{ion} concept.

\subsection{Offline Phase}

The offline phase consists of the creation of three data structures in order:

\begin{itemize}
 \item O-Space (ONION Space) is a data structure obtained by carrying out a $k$nn search and then saving the results, discarding unnecessary information.
 \item P-Space (Parameter Space) is created by sorting the knn results by the computed distance value for each value of k. This results in an ordered list of valuees to which binary-search style algorithms can be applied for more efficient outlier detection.
 \item D-Space (Domination Space) is based on a \emph{domination} relation, in which one point is said to dominate another if every parameter setting that marks it as an outlier also marks the other as an outlier. This relation typically holds among most data points and information about this relation can be encoded within a list of trees (or \emph{forest}).
\end{itemize}

It is important to note that not the entire offline phase has to be performed to allow for use of the ONION system. All of the three outlier analysis operations can be carried out using any one of these data structures, however the more of this preprocessing is performed, the faster subsequent outlier queries can be performed. Depending on the data set and parameter values, the creation of \emph{O-Space} might e.g. take much long than e.g. \emph{D-Space}, in which case it might be advisable to carry out the whole offline phase. In other cases the computation of \emph{D-Space} is going to be entirely uneconomical.

Also, \emph{P-Space} consists of the exact same data as \emph{O-Space}, which means that once \emph{P-Space} is computed, \emph{O-Space} can be discarded to free up memory.
\subsection{Online Phase}


Example citation \cite{DBLP:conf/civr/BeecksUS10}

%
% ---- Bibliography ----
%

\bibliographystyle{abbrv}
\bibliography{references}  

\end{document}
